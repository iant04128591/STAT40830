\documentclass[11pt,]{article}
\usepackage[left=1in,top=1in,right=1in,bottom=1in]{geometry}
\newcommand*{\authorfont}{\fontfamily{phv}\selectfont}
\usepackage[]{mathpazo}


  \usepackage[T1]{fontenc}
  \usepackage[utf8]{inputenc}



\usepackage{abstract}
\renewcommand{\abstractname}{}    % clear the title
\renewcommand{\absnamepos}{empty} % originally center

\renewenvironment{abstract}
 {{%
    \setlength{\leftmargin}{0mm}
    \setlength{\rightmargin}{\leftmargin}%
  }%
  \relax}
 {\endlist}

\makeatletter
\def\@maketitle{%
  \newpage
%  \null
%  \vskip 2em%
%  \begin{center}%
  \let \footnote \thanks
    {\fontsize{18}{20}\selectfont\raggedright  \setlength{\parindent}{0pt} \@title \par}%
}
%\fi
\makeatother




\setcounter{secnumdepth}{0}



\title{R Package : \emph{geegeer}  }



\author{\Large Ian Towey\vspace{0.05in} \newline\normalsize\emph{STAT40830 - Advanced R (week 6 assignment)}  }


\date{}

\usepackage{titlesec}

\titleformat*{\section}{\normalsize\bfseries}
\titleformat*{\subsection}{\normalsize\itshape}
\titleformat*{\subsubsection}{\normalsize\itshape}
\titleformat*{\paragraph}{\normalsize\itshape}
\titleformat*{\subparagraph}{\normalsize\itshape}


\usepackage{natbib}
\bibliographystyle{apsr}



\newtheorem{hypothesis}{Hypothesis}
\usepackage{setspace}

\makeatletter
\@ifpackageloaded{hyperref}{}{%
\ifxetex
  \usepackage[setpagesize=false, % page size defined by xetex
              unicode=false, % unicode breaks when used with xetex
              xetex]{hyperref}
\else
  \usepackage[unicode=true]{hyperref}
\fi
}
\@ifpackageloaded{color}{
    \PassOptionsToPackage{usenames,dvipsnames}{color}
}{%
    \usepackage[usenames,dvipsnames]{color}
}
\makeatother
\hypersetup{breaklinks=true,
            bookmarks=true,
            pdfauthor={Ian Towey (STAT40830 - Advanced R (week 6 assignment))},
             pdfkeywords = {geegeer, horse racing},  
            pdftitle={R Package : \emph{geegeer}},
            colorlinks=true,
            citecolor=blue,
            urlcolor=blue,
            linkcolor=magenta,
            pdfborder={0 0 0}}
\urlstyle{same}  % don't use monospace font for urls



\begin{document}
	
% \pagenumbering{arabic}% resets `page` counter to 1 
%
% \maketitle

{% \usefont{T1}{pnc}{m}{n}
\setlength{\parindent}{0pt}
\thispagestyle{plain}
{\fontsize{18}{20}\selectfont\raggedright 
\maketitle  % title \par  

}

{
   \vskip 13.5pt\relax \normalsize\fontsize{11}{12} 
\textbf{\authorfont Ian Towey} \hskip 15pt \emph{\small STAT40830 - Advanced R (week 6 assignment)}   

}

}







\begin{abstract}

    \hbox{\vrule height .2pt width 39.14pc}

    \vskip 8.5pt % \small 

\noindent This document provides an introduction to R package \texttt{geegeer}.
\texttt{geegeer} contains a set of predictve model for horse racing. The
main functions of the \texttt{geegeer} API are presented below. The
models in package \texttt{geegeer} are based on the analysis available
\href{http://www.ms.unimelb.edu.au/documents/thesis/AlexThesis.pdf}{here}


\vskip 8.5pt \noindent \emph{Keywords}: geegeer, horse racing \par

    \hbox{\vrule height .2pt width 39.14pc}



\end{abstract}


\vskip 6.5pt

\noindent  \section{Introduction}\label{introduction}

\texttt{geegeer} is a \texttt{R} package that implements various models
to predict the probability that a given horse in a given race will win.
The package scrapes racecard/result information off
\href{http://www.sportinglife.com/racing}{sporting.com}.

\section{API}\label{api}

The public funtions in the package are:

\begin{itemize}
\item
  listRaces
\item
  listHorses
\item
  predictProbabilities
\item
  compareModelVSp
\end{itemize}

\section{listRaces - track date}\label{listraces---track-date}

\section{listHorses - raceId}\label{listhorses---raceid}

\section{predictRaceProbabilities - returns horse name / decimal
odds}\label{predictraceprobabilities---returns-horse-name-decimal-odds}

\section{compareModelVSp - compares the predicted model probabilities v
the offical sp for that a
runner}\label{comparemodelvsp---compares-the-predicted-model-probabilities-v-the-offical-sp-for-that-a-runner}

\section{Remarks}\label{remarks}

The package can be used to compare the probabilities from the various
statistical models versus the offical SP and

\newpage
\singlespacing 
\end{document}
